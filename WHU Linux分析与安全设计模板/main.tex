\documentclass{whureport}
% =============================================
% Part 1 Edit the info
% =============================================

\def\coursename{Linux架构分析与安全设计}
\def\homeworkname{实验一\quad Linux内核基本数据结构及内存管理原理分析}

\def\study{信息安全}
\def\id{2019302140023}
\def\name{彭雨昂}
\def\teachername{王鹃教授}

\title{\homeworkname}
\date{}
% =============================================
% Part 1 Main document
% =============================================
\begin{document}
\makecover

\section{实验名称}
Linux内核基本数据结构及内存管理原理分析




\section{实验目的}
\begin{itemize}
	\item 熟悉Linux内核基本数据结构
	\item 了解内存管理的基本原理
\end{itemize}




\section{实验步骤及内容}
\subsection{第一阶段:分析Linux内核的基本数据结构}
\begin{enumerate}
	\item 分析LINUX内核文件系统实现基本数据结构,包括file struct, inode, super\_block, dentry等,描述上述数据结构之间的关系,从代码角度理解linux文件系统管理的基本原理
\end{enumerate}





\subsection{第二阶段:执行内存管理的相关命令,分析Linux内存的基本信息}
\begin{enumerate}
	\item 执行top、free、vmstat命令,查看系统内存状态
	\item 查看内存buddyinfo和slabinfo等
\end{enumerate}





\subsection{第三阶段:分析创建进程中内存分配的过程和原理}
\begin{enumerate}
	\item 分析进程创建过程中与内存相关的主要数据结构,如task\_struct, mm\_struct, vm\_area struct等,理解其含义
	\item 分析进程创建过程中内存分配的相关代码,简述其基本原理
\end{enumerate}





\section{实验关键过程及其分析}





\section{问题及思考}
\subsection{当Copy一个文件时,系统会新生成inode吗?}



\subsection{Linux中文件名和inode是如何建立联系,从而找到文件的?}





\subsection{思考内存管理数据结构中,mm\_struct、vma、vaddr、page、pfn、pte、zone、paddrr和pd\_data等的相互关系,给出你的理解。}




\end{document}